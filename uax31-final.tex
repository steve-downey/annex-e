% \infannex{uaxid}{Conformance with \UAX{31}}

\rSec1[uaxid.general]{General}

\pnum
This Annex describes the choices made in application of
\UAX{31} (``Unicode Identifier and Pattern Syntax'')
to \Cpp{} in terms of the requirements from \UAX{31} and
how they do or do not apply to this document.
In terms of \UAX{31},
this document conforms by meeting the requirements
R1 ``Default Identifiers'' and
R4 ``Equivalent Normalized Identifiers'' from \UAX{31}.
The other requirements from \UAX{31}, also listed below,
are either alternatives not taken or do not apply to this document.

\rSec1[uaxid.def]{R1 Default identifiers}

\rSec2[uaxid.def.general]{General}
\indextext{XID_Start}%
\indextext{XID_Continue}%

\pnum
\UAX{31} specifies a default syntax for identifiers
based on properties from the Unicode Character Database, \UAX{44}.
The general syntax is
\begin{outputblock}
<Identifier> := <Start> <Continue>* (<Medial> <Continue>+)*
\end{outputblock}
where \tcode{<Start>} has the XID_Start property,
\tcode{<Continue>} has the XID_Continue property, and
\tcode{<Medial>} is a list of characters permitted between continue characters.
For \Cpp{} we add the character \unicode{005f}{low line}, or \tcode{_},
to the set of permitted \tcode{<Start>} characters,
the \tcode{<Medial>} set is empty, and
the \tcode{<Continue>} characters are unmodified.
In the grammar used in \UAX{31}, this is
\begin{outputblock}
<Identifier> := <Start> <Continue>*
<Start> := XID_Start + @\textrm{\ucode{005f}}@
<Continue> := <Start> + XID_Continue
\end{outputblock}

\pnum
This is described in the \Cpp{} grammar in \ref{lex.name},
where \grammarterm{identifier} is formed from
\grammarterm{identifier-start} or
\grammarterm{identifier} followed by \grammarterm{identifier-continue}.

\rSec2[uaxid.def.rfmt]{R1a Restricted format characters}

\pnum
The clause R1a has been removed from \UAX{31}.
\begin{quote}
The characters that were added when meeting this requirement are now part of the default; the contextual checks required by this requirement remain as part of the General Security Profile in Unicode Technical Standard \#39, “Unicode Security Mechanisms”.
\end{quote}

\rSec2[uaxid.def.stable]{R1b Stable identifiers}

\pnum
An implementation of \UAX{31} may choose to guarantee
that identifiers are stable across versions of the Unicode Standard.
Once a string qualifies as an identifier it does so in all future versions.

\pnum
\Cpp{} does not make this guarantee,
except to the extent that \UAX{31} guarantees
the stability of the XID_Start and XID_Continue properties.

\rSec1[uaxid.eqn]{R4 Equivalent normalized identifiers}

\pnum
\UAX{31} requires that implementations describe
how identifiers are compared and considered equivalent.

\pnum
This document requires that identifiers be in Normalization Form C and
therefore identifiers that compare the same under NFC are equivalent.
This is described in \ref{lex.name}.

\rSec1[uaxid.nonobservance]{Requirements of \UAX{31} for which no claims are made}

\pnum
  \UAX{31} version 15.1 has conformance requirements which either do not apply to \Cpp{} or which this document makes no claim about.
  \begin{itemize}
  \item R2. Immutable Identifiers
  \item R3. Pattern_White_Space and Pattern_Syntax Characters
  \item R5. Equivalent Case-Insensitive Identifiers
  \item R6. Filtered Normalized Identifiers
  \item R7. Filtered Case-Insensitive Identifiers
  \item R8. Hashtag Identifiers
\end{itemize}

\pnum
This document also makes no claim about additional conformance points in any versions of \UAX{31} in versions after 15.1.
